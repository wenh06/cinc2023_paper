
\begin{abstract}

% almost finished

Aim: This work studies the problem of predicting neurological recovery from coma with longitudinal electroencephalogram (EEG) recordings raised by the George B. Moody PhysioNet Challenge 2023. This problem is crucial for continuous brain monitoring after cardiac arrests.

Methods: Deep neural network (DNN) models were trained to predict cerebral performance category (CPC) scale (1 to 5) from bipolar EEGs which were rescaled to zero mean and unit variance. The prediction was treated as a 5-class classification task. The models adopted a bottleneck SE-ResNet backbone with long short-term memory (LSTM) and global attention modules on its top.

Via stratified splitting on clinical attributes (age, sex, etc.) and prediction targets, 20\% of the public training data were left out as a validation set for model selection. Recordings were randomly sliced to lengths of 180 seconds every 5 epochs during training. The AdamW optimizer along with the OneCycle scheduler with maximum learning rate of 0.008 was used to optimize the model weights on the asymmetric loss of the training data.

Predictions of multiple EEG recordings from one patient were merged via manually designed rules to give final CPC and clinical outcome prediction. In cases where no EEG recording was available, random forest regressor and classifier trained using only the clinical attributes replaced DNNs to make predictions.

Results: Our team (''Revenger'')'s best entry submission received a Challenge score of 0.701 on the hidden validation set. The scores on the 12h/24h/48h subsets were 0.33, 0.4, 0.75 respectively.

Conclusion: Our solution offers a practical way to continuously monitor the brain after cardiac arrest using EEGs, with room and potential for further enhancements.

\end{abstract}
