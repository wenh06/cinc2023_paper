\begin{abstract}

% finished

Aim: This work studies the problem of predicting neurological recovery from coma with longitudinal electroencephalogram (EEG) recordings raised by the George B. Moody PhysioNet Challenge 2023.

Methods: Deep neural network (DNN) models were trained to predict cerebral performance category (CPC) scale (1 to 5) from bipolar EEGs which were rescaled to zero mean and unit variance. The prediction was treated as a 5-class classification task. The models adopted a bottleneck SE-ResNet backbone with long short-term memory (LSTM) and global attention modules on its top.

Via a stratified splitting, 20\% of the training data were left out as a validation set for model selection. Recordings were chosen and sliced according to the pre-computed signal quality index for training. The \texttt{AdamW} optimizer and the \texttt{OneCycle} scheduler were used to optimize model weights on the asymmetric loss of the training data. Predictions of multiple EEG recordings from one patient were merged via voting strategies to give a final prediction.

Results: Our team's (''Revenger'') best submission entry received a challenge score of 0.554, ranked 12th, for the clinical outcome prediction on the hidden test set.

Conclusion: Our solution offers a practical way to continuously monitor the brain after cardiac arrest using EEGs, with room and potential for further enhancements.

\end{abstract}
