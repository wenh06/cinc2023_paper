\section{Discussion and Conclusions}
\label{sec:discu}

% almost finished.

As the results presented in Section \ref{sec:results} indicate, the TiCRNN model proposed in this work provides an effective solution to the problem of predicting the level of neurological recovery for comatose patients after cardiac arrest raised by the Challenge. It is relatively simple and lightweight, but still able to attain a TPR as high as 0.701 with FPR, which is vital for the patients in this problem, suppressed to a very low level ($\le 0.05$) for poor clinical outcome prediction. Another phenomenon that can be observed from Table \ref{tab:challenge_scores} is that the score on the hidden validation set is much higher than the score on the cross-validation set. This indicates that by combining results from multiple recordings, one is able to obtain a much more reliable prediction than from only one recording.

As introduced in Section \ref{subsec:data_selection} and \ref{subsec:training}, we made trade-offs for the limited computation resources by dropping a large proportion of the EEG data. The data amount (counting by time length) we really used for training the NN models merely constitutes 6 \% of the total EEG data. The potential of this extraordinarily big data had not yet been fully explored. Our team had planned to make full use of the data to train a larger model that learns latent representations from EEGs via unsupervised contrastive learning. However, due to the constraints of time and computation resources our team owns, we finally decided to stick to the TiCRNN model developed in the unofficial phase of the Challenge. Architecture design and unsupervised training mechanisms for large EEG models are left as future research directions.

The computation of SQI for EEGs as described in Section \ref{subsec:data_selection} is time-consuming. This prohibited us from adding a similar selection procedure in the pipeline of model evaluation on the hidden data and is highly probable to have a negative influence on our overall performance, especially for EEGs heavily contaminated with artifacts. Therefore, developing a faster and end-to-end SQI computation method would also be a meaningful research problem.

Despite the large EEG data amount, another nice feature of the I-CARE database is that it provides simultaneous electrocardiogram (ECG) signals along with the EEGs. This makes it possible to explore and develop multi-modal solutions to the Challenge problem, which is also a topic worth researching but not yet studied in this work.
